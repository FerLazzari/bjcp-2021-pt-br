\clearpage
\phantomsection
\divisorLine
\section*{Apêndice B: Estilos Locais}
\addcontentsline{toc}{section}{Apêndice B: Estilos Locais}

\textit{Este apêndice contém descrições de estilos enviados por juízes BJCP locais ou cervejeiros de um país ou região, para estilos emergentes de importância local, que podem ou não ter uma popularidade mais ampla. O BJCP revisou, editou e verificou as descrições dos estilos, que podem ser usadas por qualquer pessoa, não apenas como um apelo da local (embora juízes fora da área podem não estar familiarizados com eles). Esses estilos locais fazem parte do Guia de Estilos e não são estilos provisórios.}

\begin{multicols}{2}

  \textbf{\Large\color{blue}
  Estilos Brasileiros
  }
  \import{./appendix}{x1-dorada-pampeana.tex}
	\import{./appendix}{x2-ipa-argenta.tex}
	\import{./appendix}{x3-italian-grape-ale.tex}
	\import{./appendix}{x4-catharina-sour.tex}
	\import{./appendix}{x5-new-zealand-pilsner.tex}

\end{multicols}
