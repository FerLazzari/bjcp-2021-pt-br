\clearpage
\subsection*{Referências de Tags de Estilo}
\addcontentsline{toc}{subsection}{Referências de Tags de Estilo}

Para ajudar no reagrupamento de estilos para outras finalidades, adicionamos tags/etiquetas informacionais para cada estilo. Estas tags indicam alguns atributos da cerveja que podem ser usados para agrupamentos. A coluna “significado” explica a finalidade geral da tag, mas não é feita para ser uma definição rigorosa ou formal. De forma alguma as tags substituem as descrições reais do estilo.

\renewcommand{\arraystretch}{1.2}

\definecolor{darkgray}{rgb}{0.5, 0.5, 0.5}
\definecolor{lightgray}{rgb}{0.75, 0.75, 0.75}

\begin{longtable}{ | p{32mm} | p{32mm} | p{32mm} | p{32mm} | p{32mm} | }
\hline
\rowcolor{lightgray}
\color{white}\textbf{Categoria} & \color{white}\textbf{Tag} & \color{white}\textbf{Etiqueta} & \color{white}\textbf{Meaning} & \color{white}\textbf{Significado} \\
\endhead
\hline
\rowcolor{darkgray}
\multicolumn{5}{|l|}{\color{white}\textbf{Strength (Teor Alcoólico)}} \\
\hline
& session-strength & Teor-alcoólico-leve & <4\% ABV & <4\% vol. \\
\hline
& standard-strength & Teor-alcoólico-padrão & 4-6\% ABV & 4-6\% vol. \\
\hline
& high-strength & Teor-alcoólico-alto & 6-9\% ABV & 6-9\% vol. \\
\hline
& very-high-strength & Teor-alcoólico-muito-alto & >9\% ABV & >9\% vol. \\
\hline
\rowcolor{darkgray}
\multicolumn{5}{|l|}{\color{white}\textbf{Color (Cor)}} \\
\hline
& pale-color & Cor-clara & straw to gold & Palha a dourado \\
\hline
& amber-color & Cor-âmbar & amber to copper-brown & Âmbar a cobre amarronzado \\
\hline
& dark-color & Cor-escura & dark brown to black & Marrom escuro a preto \\
\hline
\rowcolor{darkgray}
\multicolumn{5}{|l|}{\color{white}\textbf{Fermentation (Fermenteção)/Conditioning (Maturação)}} \\
\hline
& top-fermented & Alta-fermentação & ale yeast & Levedura tipo ale \\
\hline
& bottom-fermented & Baixa-fermentação & lager yeast & Levedura tipo lager \\
\hline
& any-fermentation & Qualquer-fermentação & ale yeast or lager yeast & Levedura ale ou lager \\
\hline
& wild-fermented & Fermentação-selvagem & non-Saccharomyces yeast/bacteria & Levedura não-saccharomyces / bactérias \\
\hline
& lagered & Condicionada-a-frio & cold conditioned & Condicionamento a frio \\
\hline
& aged & Maturada & long conditioning before release & Maturação longa \\
\hline
\rowcolor{darkgray}
\multicolumn{5}{|l|}{\color{white}\textbf{Region of Origin (Região de Origem})}\\
\hline
& british-isles & Ilhas-Britânicas & England, Wales, Scotland, Ireland & Inglaterra, País de Gales, Escócia, Irlanda \\
\hline
& western-europe & Europa-Oriental & Belgium, France, Netherlands & Bélgica, França, Holanda \\
\hline
& central-europe & Europa-Central & Germany, Austria, Czech Republic, Scandinavia & Alemanha, Áustria, República Tcheca, Escandinávia \\
\hline
& eastern-europe & Europa-Ocidental & Poland, Baltic States, Russia & Polônia, Estados Bálticos, Rússia \\
\hline
& north-america & América-do-Norte & United States, Canada, Mexico & Estados Unidos, Canadá, México \\
\hline
& south-america & América-do-Sul & Argentina, Brazil & Argentina, Brasil \\
\hline
& pacific & Pacífico & Australia, New Zealand & Austrália, Nova Zelândia \\
\hline
\rowcolor{darkgray}
\multicolumn{5}{|l|}{\color{white}\textbf{Style Family (Família de Estilo)}} \\
\hline
& ipa-family & Família-das-IPA & & \\
\hline
& brown-ale-family & Família-das-brown-ale & & \\
\hline
& pale-ale-family & Família-das-pale-ales & & \\
\hline
& pale-lager-family & Família-das-pale-lagers & & \\
\hline
& pilsner-family & Família-pilsners & & \\
\hline
& amber-ale-family & Família-das-amber-ales & & \\
\hline
& amber-lager-family & Família-das-amber-lagers & & \\
\hline
& dark-lager-family & Família-das-dark-lagers & & \\
\hline
& porter-family & Família-das-porters & & \\
\hline
& stout-family & Família-das-stouts & & \\
\hline
& bock-family & Família-das-bocks & & \\
\hline
& strong-ale-family & Família-das-Strong-ales & & \\
\hline
& wheat-beer-family & Família-das-wheat-beer & & \\
\hline
& specialty-beer & Família-das-cervejas-de especialidade & & \\
\hline
\rowcolor{darkgray}
\multicolumn{5}{|l|}{\color{white}\textbf{Era (Era)}} \\
\hline
& craft-style & Cerveja-Artesanal & developed in the modern craft beer era & desenvolvido na era da cerveja artesanal moderna \\
\hline
& traditional-style & Cerveja-Tradicional & developed before the modern craft beer era & desenvolvido antes da era da cerveja artesanal moderna \\
\hline
& historical-style & Cerveja-Histórica & no longer made, or very limited production & não é mais fabricado ou uma produção muito limitada \\
\hline
\rowcolor{darkgray}
\multicolumn{5}{|l|}{\color{white}\textbf{Dominant Flavor (Sabor Dominante)}} \\
\hline
& malty & Maltado & malt-forward flavor & Malte em primeiro plano \\
\hline
& bitter & Amargo & bitter-forward flavor & Amargor em primeiro plano \\
\hline
& balanced & Equilibrado & similar malt and bitter intensity & Equilíbrio na intensidade de malte e amargor \\
\hline
& hoppy & Lupulado & hop flavor & Sabor de lúpulo \\
\hline
& roasty & Tostado & roasted malt or grain & Malte ou grãos tostados \\
\hline
& sweet & Adocicado & noticeable residual sweetness or sugar flavor & Residual doce aparente ou sabor de açúcar \\
\hline
& smoke & Defumado & flavor of smoked malt or grain & Sabor de malte ou grãos defumado \\
\hline
& sour & Ácido/Azedo & noticeable sourness or intentionally elevated acidity & Acidez aparente ou acidez intencionalmente elevada \\
\hline
& wood & Amadeirado & wood or barrel age character & Característica de madeira ou maturação em barril \\
\hline
& fruit & Frutado & noticeable flavor or aroma of fruit & Aroma ou sabor aparente de fruta \\
\hline
& spice & Condimentado & noticeable flavor or aroma of spices & Aroma ou sabor aparente de especiarias \\
\hline
\end{longtable}