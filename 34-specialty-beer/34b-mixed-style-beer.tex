\phantomsection
\subsection*{34B. Mixed-Style Beer}
\addcontentsline{toc}{subsection}{34B. Mixed-Style Beer}

\textit{Este estilo se destina a cervejas em \textbf{Estilos Existentes} (definidos previamente em Estilos Clássicos e Cervejas de Especialidade), e pode ser:}

\begin{itemize}
  \item Uma combinação de Estilos Existentes que não estão definidos anteriormente nas diretrizes, incluindo um combinação de Cervejas de Especialidade que não se enquadram em outras categorias.
  \item Uma variação de um Estilo Existente, que usa um método ou processo não tradicional para esse estilo (por exemplo, dry-hopping, stein bier, 'eis'-ing *(processo de obtenção de álcool por congelamento da cerveja, como na Eisbock)*).
  \item Uma variação de um Estilo Existente, que usa um ingrediente não tradicional (por exemplo, levedura com um perfil não tradicional, lúpulo com um caráter diferente do descrito no Estilo Base).
  \item Variações fora de especificação de um Estilo Existente (por exemplo, versões "Imperial", "Session", excessivamente doces, etc).
\end{itemize}

\textit{Este estilo é destinado às cervejas que não se enquadram nos estilos listados anteriormente, incluindo o Estilo Base declarado. No entanto, se o método, processo ou ingrediente incomum resultarem em uma cerveja que se enquadre em um estilo já definido, a cerveja deve ser lá inscrita. Observe que alguns estilos permitem diferentes intensidades (por exemplo, IPAs, Saisons), portanto, essas variações devem ser inscritas no Estilo Base apropriado.}

\textit{Tenha em mente que uma cerveja mal executada e com defeito não deve ser usada para definir um novo estilo. A facilidade de bebê-la deve sempre ser mantida, enquanto permite o uso da criatividade em novos conceitos.}

\textbf{Impressão Geral}: De acordo com os estilos base, métodos e ingredientes declarados. Assim como todas as Cervejas de Especialidade, o resultado da combinação dos estilos precisa ser harmonioso, equilibrado e agradável de beber.

\textbf{Aroma / Aparência / Sabor / Sensação na Boca}: De acordo com os estilos base declarados.

\textbf{Comentários}: Veja o preâmbulo para o propósito.

\textbf{Instruções para Inscrição}: O participante deve especificar o Estilo ou Estilos Base que estão sendo usados e quaisquer ingredientes, processos ou variações especiais envolvidos. O participante deve fornecer uma descrição adicional do perfil sensorial da cerveja e/ou as estatísticas da cerveja final.

\textbf{Estatísticas}: OG, FG, IBU, SRM e ABV vão variar de acordo com a cerveja declarada.

\textbf{Exemplos Comerciais}: Birrificio Italiano Tipopils, Firestone Walker Pivo Pils, Jack’s Abby Hoponius Union, Ommegang Helles Superior.

\textbf{Última Revisão}: Mixed-Style Beer (2015)

\textbf{Atributos de Estilo}: specialty-beer
