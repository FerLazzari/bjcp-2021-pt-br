\phantomsection
\subsection*{10B. Dunkles Weissbier}
\addcontentsline{toc}{subsection}{10B. Dunkles Weissbier}

\textbf{Impressão Geral}: Uma cerveja de trigo alemã moderadamente escura com um característico perfil de fermentação de banana e cravo da levedura weizen, apoiado por sabor de malte como pão tostado e caramelo. Altamente carbonatada e refrescante com uma textura cremosa e macia e um final leve.

\textbf{Aparência}: De cor cobre clara a marrom mogno escuro. Colarinho quase branco muito denso, como mousse e duradouro. Pode ser turva e ter um brilho do trigo e levedura, embora isso possa se depositar no fundo da garrafa.

\textbf{Aroma}: Ésteres e fenóis moderados, normalmente banana e cravo, geralmente bem equilibrado entre os dois e o malte. Aroma de baixo a moderado como pão, como massa de pão ou como cereal de trigo, muitas vezes acompanhada por notas de caramelo, casca de pão ou notas ricas maltadas. Baunilha de baixo a moderado é opcional. Lúpulo floral, condimentado e/ou herbal de intensidade leve é opcional. Bubblegum/Tuttifrutti (morango com banana), acidez e defumado são falhas.

\textbf{Sabor}: Sabor de banana e cravo de baixo a moderadamente forte, geralmente bem equilibrado entre os dois e o malte, embora o malte às vezes possa mascarar a impressão de cravo. Sabor macio de pão, massa de pão ou de cereal lembrando trigo de baixo a médio-alto, com sabores ricos de caramelo, tostado ou casca de pão. Sem sabores fortes de torra, mas um toque de secura derivada do torrado é permitido. Amargor de baixo a muito baixo. Bem equilibrada, saborosa, geralmente na boca é um tanto maltada com um final relativamente seco. Baunilha de muito leve a moderado é opcional. Sabor de lúpulo condimentado, herbal e/ou floral baixo é opcional. Bubblegum/Tuttifrutti (morango com banana), acidez e defumado são falhas.

\textbf{Sensação na Boca}: Corpo de médio-leve a médio-cheio. Cremosidade macia e cheia evoluindo para um final mais leve com a ajuda de uma carbonatação de moderada a alta. Efervescente.

\textbf{Comentários}: Muitas vezes conhecida como \textit{dunkelweizen}, principalmente nos Estados Unidos. Cada vez mais rara e muitas vezes sendo substituída pelas versões \textit{Kristall} e não alcoólicas na Alemanha.

\textbf{História}: A Baviera tem uma tradição secular de produção de cervejas de trigo, mas os direitos de produção eram reservados para a realeza bávara até o final do século 18. A cerveja bávara de trigo a moda antiga era normalmente escura, assim como a grande maioria das cervejas daquele tempo. Weissbiers claras começaram a se tornar populares na década de 1960, mas a cerveja de trigo escura tradicional se manteve como uma bebida para idosos.

\textbf{Ingredientes}: Malte de trigo, pelo menos metade do perfil de maltes. Malte Munich, Vienna e/ou Pilsner. Malte de trigo escuro, malte caramelo de trigo e/ou malte de cor. Mostura por decocção é tradicional. Levedura weizen e temperaturas mais baixas de fermentação.

\textbf{Comparação de Estilo}: Combina o caráter de levedura e de trigo de uma Weissbier com a riqueza maltada de uma Munich Dunkel. O perfil de banana e cravo é muitas vezes menos aparente do que em uma Weissbier devido ao elevado caráter maltado. Apresenta característica de levedura similar a uma Roggenbier, mas sem o sabor de centeio e o corpo mais cheio.

\begin{tabular}{@{}p{35mm}p{35mm}@{}}
  \textbf{Estatísticas} & OG: 1,044 - 1,057 \\
  IBU: 10 - 18 & FG: 1,008 - 1,014 \\
  SRM: 14 - 23 & ABV: 4,3\% - 5,6\%
\end{tabular}

\textbf{Exemplos Comerciais}: Ayinger Ur-Weisse, Franziskaner Hefe-Weisse Dunkel, Ettaler Benediktiner Weißbier Dunkel, Hirsch Dunkel Weisse, Tucher Dunkles Hefe Weizen, Weihenstephaner Hefeweissbier Dunkel.

\textbf{Última Revisão}: Dunkles Weissbier (2015)

\textbf{Atributos de Estilo}: amber-color, central-europe, malty, standard-strength, top-fermented, traditional-style, wheat-beer-family
