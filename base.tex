\begin{multicols}{2}
[
\section*{1. Standard American Beer}
Esta categoria descrever cervejas americanas de todos os dias, de grande aceitação do público. Contendo ales e lagers, as cervejas desta categoria tipicamente não são complexas e possuem sabores suaves e fáceis de beber. As ales normalmente têm características semelhantes às lagers e são projetadas para capturars os bebedores de lagers de massa, como cervejas de transição. Cervejas de consumo popular, de sabor mais internacional ou de origem internacional, são descritas na categoria Internacional Lager.
]

\subsection*{1A. American Light Lager}

\subsubsection*{Impressões gerais}
Altamente carbonatada e de corpo muito baixo, lager quase sem sabor, criada para ser consumida bem gelada. Muito refrescante e para matar a sede.

\subsubsection*{Aparência}

Cor de palha ao amarelo claro. Espuma branca não muito persistente. Límpida.

\subsubsection*{Aroma}

De baixo a nenhum aroma de malte, caso presente, pode ser percebido como grãos, doce ou milho. Aroma de lúpulo de baixo a nenhum, podendo ter notas de picante, floral ou herbal. Apesar de ser desejável um caráter limpo de fermentação, uma leve característica de levedura não é uma falha.

\subsubsection*{Sabor}

Relativamente neutra ao palato com um final fresco e seco. Sabor de grãos ou milho de baixo a muito baixo, que pode ser percebido como doçura devido ao baixo amargor. Sabor de lúpulo de baixo a nenhum, podendo ter perfil floral, picante ou herbal, embora seja raramente forte para ser detectado. Amargor de lúpulo de baixo a muito baixo. Equilíbrio pode variar de ligeiramente maltado ao ligeiramente amargo, mas comumente equilibrada. A alta carbonatação pode realçar a sensação de frescor e o final seco. Caráter limpo de fermentação lager.

\subsubsection*{Sensação na Boca}

Corpo muito leve, às vezes aguado. Carbonatação muito alta com carbonatação picante na língua.

\subsubsection*{Comentários}

Desenvolvida para cativar a mais ampla gama de pessoas possível. Sabores fortes significam falha na cerveja. Com pouco sabor de malte ou lúpulo, a característica da levedura muitas vezes é o que diferencia as marcas.

\subsubsection*{História}

A Coors produziu uma Light Lager por alguns anos na década de 1940. Versões modernas foram produzidas inicialmente por Rheingold em 1967 para atender os consumidores que faziam dieta, mas somente em 1973 se tornou popular, após a cervejaria Miller adquirir a receita e fazer grande propaganda de \textit{marketing} entre os esportistas com o slogan "muito gosto, menos calorias". As cervejas deste estilo se tornaram as mais vendidas nos EUA na década de 1990.

\subsubsection*{Ingredientes Característicos}

Cevada de duas ou seis fileiras com até 40\% de adjuntos (arroz ou milho). Enzimas adicionais podem ser utilizadas para reduzir o corpo e a quantidade de carboidratos. Levedura Lager. Pouco uso de lúpulos.

\subsubsection*{Comparação de Estilos}

Uma versão com menor corpo, menos álcool e menos calorias do que uma American Lager. Menos caráter de lúpulo e amargor do que na German Leitchbier.

\begin{tabular}{@{}ll@{}}
\subsubsection*{Estatística Vital} & OG: 1.028 - 1.040 \\
IBU: 8 - 12 & FG: 0.998 - 1.008 \\
SRM: 2 - 3  & ABV: 2.8\% - 4.2\%
\end{tabular}

\subsubsection*{Exemplos Comerciais}

Bud Light, Coors Light, Grain Belt Premium Light American Lager, Michelob Light, Miller Lite, Old Milwaukee Light.

\subsubsection*{Revisões Anteriores}

American Light Lager (2015)

\subsubsection*{Atributos do Estilo}

Balanço, baixa-fermentação, Lagered (maturada), América do Norte, Cor pálida, família-pale-lager, Intensidade session, estilo tradicional.

\vspace{5mm}

\subsection*{1B. American Lager}

\subsubsection*{Impressões gerais}

Uma cerveja lager muito clara, altamente carbonatada, de corpo baixo, bem atenuada com um sabor neutro e baixo amargor. Servida bem gelada, muito refrescante e para matar a sede.

\subsubsection*{Aparência}

Cor de palha ao amarelo médio. Espuma branca não muito persistente. Límpida.

\subsubsection*{Aroma}

De baixo a nenhum aroma de malte, caso presente, pode ser percebido como grãos, doce ou milho. Aroma de lúpulo de baixo a nenhum, podendo ter notas picante, floral ou herbal. Apesar de ser desejável um caráter limpo de fermentação, uma leve característica de levedura não é uma falha.

\subsubsection*{Sabor}

Relativamente neutra ao palato com um final fresco e seco. Sabor de grão ou milho de baixo a médio-baixo, que pode ser percebido como doçura devido ao baixo amargor. Sabor de lúpulo de baixo a nenhum, podendo ter perfil floral, picante ou herbal, embora seja raramente forte para ser detectado. Amargor de lúpulo de baixo a médio-baixo. Equilíbrio pode variar de ligeiramente maltado ao ligeiramente amargo, mas comumente equilibrada. A alta carbonatação pode realçar a sensação de frescor e o final seco. Caráter limpo de fermentação lager.

\subsubsection*{Sensação na Boca}

Corpo de baixo a médio-baixo. Carbonatação muito alta cria a sensação de leve carbonatação picante na língua.

\subsubsection*{Comentários}

Cerveja esperada pelos bebedores de cerveja não artesanal no Estados Unidos. Pode ser comercializada como Pilsner fora da Europa, mas não deve ser confundida com os exemplos tradicionais. Sabores fortes significam falha na cerveja. Com pouco sabor de malte ou lúpulo, a característica da levedura muitas vezes é o que diferencia as marcas.

\subsubsection*{História}

Evolução da Pre-Prohibition Lager (ver Categoria 27) nos Estados Unidos após a Lei Seca e a Segunda Guerra Mundial. Cervejarias sobreviventes se consolidaram, expandiram a distribuição e promoveram intensamente um estilo de cerveja que satisfazia grande parte da população. Se tornou o estilo dominante por muitas décadas e, por isto, surgiram muitos rivais internacionais que desenvolveram produtos semelhantes para o mercado de massa utilizando campanhas publicitárias agressivas.

\subsubsection*{Ingredientes Característicos}

Cevada de duas ou seis fileiras com até 40\% de adjuntos (arroz ou milho). Levedura Lager. Pouco uso de lúpulos.

\subsubsection*{Comparação de Estilos}

Uma versão mais forte, com mais sabor e corpo do que uma American Light Lager. Menos amargor e sabor do que uma International Pale Lager. Com muito menos sabor, lúpulo e amargor do que as tradicionais Pilsners européias.

\subsubsection*{Estatística Vital}

\begin{tabular}{@{}ll@{}}
\subsubsection*{Estatística Vital} & OG: 1.040 - 1.050 \\
IBU: 8 - 18 & FG: 1.004 - 1.010 \\
SRM: 2 - 3.5 & ABV: 4.2\% - 5.3\%
\end{tabular}

\subsubsection*{Exemplos Comerciais}

Budweiser, Coors Original, Grain Belt Premium Lager, Miller High Life, Old Style, Pabst Blue Ribbon, Special Export.

\subsubsection*{Revisões Anteriores}
American Lager (2015)

\subsubsection*{Atributos do Estilo}

Balanço, baixa-fermentação, Lagered (maturada), América do Norte, Cor pálida, família-pale-lager, Intensidade padrão, estilo tradicional.

\end{multicols}