\section*{14. Scottish Ale}
\addcontentsline{toc}{section}{14. Scottish Ale}

Existem realmente apenas três estilos tradicionais de cerveja amplamente disponíveis hoje na Escócia: a 70/- Scottish Heavy, a 80/- Scottish Export e a Strong Scotch Ale (Wee Heavy, Estilo 17C). A 60/- Scottish Light é rara e muitas vezes apenas em barris, mas parece estar tendo um tímido renascimento atualmente. Todos esses estilos tomaram forma moderna após a Segunda Guerra Mundial, independentemente de usarem os mesmos nomes anteriormente. Atualmente, o 60/- é semelhante a dark mild, o 70/- é semelhante a ordinary bitter e o 80/- semelhante a best ou strong bitter. As cervejas escocesas têm um perfil de equilíbrio e sabor diferente, mas ocupam uma posição de mercado semelhante às cervejas inglesas. As cervejas Light, Heavy e Export têm perfis de sabor semelhantes, e muitas vezes são produzidas pelo método parti-gyle. À medida que a densidade aumenta, o mesmo acontece com o caráter da cerveja. Os ingredientes tradicionais eram malte pale ale com alta dextrina, milho, açúcares escuros e caramelo cervejeiro para dar cor. As receitas modernas (pós-Segunda Guerra Mundial) geralmente adicionam pequenas quantidades de malte dark e porcentagens mais baixas de malte crystal, juntamente com outros ingredientes como malte âmbar e trigo. Os cervejeiros escoceses tradicionalmente usavam mosturas de infusão única, muitas vezes mosturas com a técnica underlet e múltiplas lavagens. Em geral, essas cervejas escocesas são mais fracas, mais doces, mais escuras, com menor atenuação e menos lupuladas em comparação com as cervejas inglesas modernas equivalentes. Elas são produzidas usando temperaturas de fermentação ligeiramente mais frias do que suas correspondentes inglesas. Muitas dessas diferenças foram exageradas na cultura popular; eles são perceptíveis, mas não enormes, mas o suficiente para afetar o equilíbrio da cerveja e talvez indicar uma preferência nacional de sabor. O equilíbrio permanece maltado e um tanto adocicado devido à maior densidade final, menor teor alcoólico e menores lupulagens. Muitas dessas divergências com a cerveja inglesa ocorreram entre o final de 1800 e meados de 1900. Métodos de produção defendidos por cervejeiros caseiros, como caramelização por fervura ou conjunto de grãos carregado em uma variedade de maltes crystal, não são comumente usados em produtos tradicionais, mas podem aproximar esses sabores quando os ingredientes tradicionais não estão disponíveis. O uso de malte defumado em fogo de turfa não é apenas completamente inautêntico, mas também produz um sabor poluído e fenólico inadequado em qualquer um desses estilos. As versões defumadas (usando qualquer tipo de defumação) devem ser inseridas em 32A Classic Style Smoked Beer. O uso de designações 'shilling' (/-) é uma curiosidade escocesa. Originalmente, referia-se ao preço da cerveja em barris, que de forma alguma poderia ser constante ao longo do tempo. Xelins nem são usados como moeda agora na Escócia. Mas o nome ficou como uma abreviação de um tipo de cerveja, mesmo que o significado original tenha deixado de ser o preço real durante a Primeira Guerra Mundial. Tudo o que isso significa agora é que números maiores designam cervejas mais fortes, pelo menos dentro da mesma cervejaria. Entre as guerras mundiais, algumas cervejarias usavam o preço por pint em vez de xelins (por exemplo, Maclay 6d por 60/-, 7d por 70/-, 8d por 80/-). Confusamente, durante esse período, a 90/- pale ale era uma cerveja engarrafada de baixa densidade. Curioso, de fato.
