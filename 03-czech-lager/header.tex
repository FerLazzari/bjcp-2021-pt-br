\section*{3. Czech Lager}
\addcontentsline{toc}{section}{3. Czech Lager}

Lagers tchecas são geralmente divididas por classe de densidade (draft, lager, especial) e cor (clara, âmbar, escura). Os nomes tchecos para essas categorias são \textit{svetlé} (clara), \textit{polotmavé} (âmbar) e \textit{tmavé} (escura). As classes de densidade são \textit{výcepní} (draft 7–10 °P), \textit{ležák} (lager 11–12 °P) e \textit{speciální} (especial 13 °P+). Pivo é a palavra tcheca para cerveja. A divisão em classes de densidade é similar ao agrupamento alemão em \textit{schankbier}, \textit{vollbier}, e \textit{starkbier}, embora a extensão de densidade seja diferente. As cervejas tchecas dentro das classes são geralmente referenciadas apenas pela sua densidade. Geralmente existem variações entre os agrupamentos de densidade-cor, principalmente na classe \textit{speciální}. O guia de estilos combina algumas dessas classes, enquanto outras cervejas do mercado tcheco não são descritas (tal como a Czech Strong Porter). Isso não significa que as categorias abaixo abrangem as cervejas tchecas em sua totalidade, é simplesmente uma maneira de agrupar alguns dos exemplares mais comumente encontrados para propósitos de julgamento. Lagers tchecas são geralmente diferenciadas das alemãs e outras lagers ocidentais no aspecto de que as lagers alemãs são normalmente totalmente atenuadas, enquanto as lagers tchecas podem conter uma pequena porção de extrato não fermentado permanecendo na cerveja pronta. Como consequência, a densidade final é levemente maior (com menor atenuação aparente), corpo e sensação de boca levemente maiores e um perfil de sabor mais rico e complexo em cervejas de cor e força equivalentes. Lagers alemãs tendem a apresentar um perfil de fermentação mais limpo, enquanto as lagers tchecas são fermentadas mais frias (7–10°C) e por um período maior de tempo e podem ter uma leve, quase imperceptível (perto do limiar de percepção) quantidade de Diacetil que geralmente é percebido mais como um corpo aredondado do que aparente no aroma e sabor (aroma de manteiga aparente é uma falha). As cepas de leveduras lager tchecas não são sempre tão limpas e atenuantes como as cepas alemãs, o que ajuda a atingir uma densidade final maior (juntamente com os métodos de mosturação e fermentação mais fria). Lagers tchecas são tradicionalmente produzidas com decocção (geralmente dupla), mesmo com maltes modernos, enquanto a maioria das lagers alemãs são produzidas com mosturação por infusão única ou em etapas. Essas diferenças caracterizam o perfil rico, a sensação de boca e o perfil de sabor que distinguem as lagers tchecas.
